\chapter{Specification and system design}
\label{spec}
\section{Data collection}
The quantity and the quality of training data is crucial to all computer vision 
projects. However, for this task there was no luxury of using an existing 
dataset with the required level of detail in labels that would also be 
affordable.

A decision was made to collect own data for this project from a social network 
that makes it easy to associate user profiles with photos of themselves with 
locations accurate to city-level. Popular social networks such as Facebook 
would be ideal for this, however it is not possible to obtain a list of 
profiles from a hand-picked location due to Facebook API constraints. 
Moreover, it is clearly stated in Facebook's Terms and Conditions that any way 
to circumvent these constraints would still count as a violation.


\subsection{Tinder}
Tinder is a mobile dating application that shows profiles closest to the 
user's GPS location. There have been numerous third-party Tinder utilities on 
the Web that reverse-engineered Tinder's simple and unobfuscated HTTP API that 
makes it possible to create a fully-featured Tinder client yourself.

The API has support for various actions such as liking and messaging users, 
but for the purposes of this project, the only relevant actions to us are 
authentication, setting own location using latitude and longitude coordinates 
and fetching nearby users.

Tinder profiles are created using Facebook Login. When the application is 
opened for the first time by a non-registered user, they will be prompted to 
login using their Facebook account which gives Tinder access to the user's full name, 
age, pictures and other details.

A user profile was created specifically for this project using the original 
Tinder mobile application on an Android phone. 

\subsection{Profile metadata and photo collection}
A web application (\texttt{tinder-gather}) was written to mimic the Tinder app whose only purpose was 
to record profile metadata such as name, location and date birth as well as 
to save user-uploaded pictures. 

This web application provides an API that allows an external tool 
(\ref{spec:data:jobs}) to easily change the location of the user for this 
application so that images are collected from several different 
countries.

The web application must be used with a complementary Chrome extension 
(\texttt{tinder-gather-connect}) that handles Facebook authentication.
Once the extension is installed a button to launch the app will appear in the 
toolbar. As soon as the user signs in the app will be ready to accept data 
collection jobs from the Tinder job manager tool (\ref{spec:data:jobs}).

\subsection{Tinder job manager}
\label{spec:data:jobs}
Images of people need to be collected from a number of different locations 
around the world. Specifying the coordinates of each location using latitude 
and longitude values proved to be cumbersome in the beginning which is why a 
special and user-friendly tool was written to manage data collection from the 
command line.

This tool resides in the \texttt{tinder-gather} project in the \texttt{tools} 
directory. Usage and help will be displayed in the terminal if ran with these 
arguments:
\begin{logs}
./jobs.js tjob post
\end{logs}

This command expects three arguments: location, limit and delay. Location is 
just a city name whose location is hard-coded inside the script and can be 
easily extended. This argument is required. Limit refers to the maximum number 
of profiles to fetch from Tinder at once and delay specifies how frequently 
they will be fetched.

This tool is also used to manage face and landmark detection jobs which will 
be described in detail in Section \ref{spec:fd}. 

\section{Face and landmark detection}
\label{spec:fd}
Before the collected pictures of faces from different parts of the world 
could be used to learn how to determine the location of a person from a single 
picture of their face they must first be heavily filtered as many Tinder users 
upload very low-quality pictures or even pictures of pets and inanimate 
objects.

A face detector is an ideal method of filtering out non-face images as well as
heavily edited or poor quality face pictures. 

Viola-Jones is the most popular choice for face detection due to its 
incredible performance and native implementation in OpenCV, but it falls 
behind some other face detection algorithms such as those used by 
Google Picasa and face.com.

Unfortunately, Google Picasa and face.com use closed-source 
solutions making them unsuitable for use in this project. However, it was 
discovered that there is an open face detection model that comes very close 
to these commercial algorithms and significantly outperforms Viola-Jones at
the face detection task.

The model can also be trained to learn landmark localisation and face angle, 
which provide additional information that enables us to further filter and 
preprocess the images collected from Tinder profiles. This is important 
because being able to estimate the locations of certain landmarks such as eyes 
makes it possible to scale and rotate images so that the landmarks line up.

Another advantage of this particular solution is that it comes with 
pre-trained on the MultiPIE dataset which otherwise would need to be purchased. 

\subsection{Face, landmark and pose detection using mixture of trees and HoG}
In this model, there is a shared pool of facial landmarks encoded as parts and denoted $V$.
Every viewpoint is modelled as a tree-structured pictorial structure $T_m = (V_m, E_m)$
where $E_m$ is a set of edges and $V_m \in V$.

The locations of each part ($L = {l_i : i \in V}$) can be scored as follows:
\begin{flalign}
    \label{eq:spec:fd:S}
    S(I,L,m) = \text{App}_m(I,L) + \text{Shape}_m(L) + \alpha^m 
\end{flalign}

\begin{flalign}
    \label{eq:spec:fd:App}
    \text{App}_m(I,L) = \sum_{i \in V_m} w_i^m \cdot \phi(I,l_i) 
\end{flalign}

\begin{flalign}
    \label{eq:spec:fd:Shape}
    \text{Shape}_m(L) = \sum_{ij \in E_m} a_{ij}^m dx^2 + b_{ij}^m dx + c_{ij}^mdy^2 + d_{ij}^m dy 
\end{flalign}

Equation \ref{eq:spec:fd:App} computes the appearance evidence of part $i$
using template $w_i^m$ which is tuned for mixture/viewpoint $m$. $\phi(I,l_i)$
is the feature vector (e.g. Histogram of oriented gradients) at location $l_i$
of image $I$.

In Equation \ref{eq:spec:fd:Shape} $dx = x_i  - x_j$ and $dy = y_i - y_j$
correspond to $x$ and $y$ displacements of part $i$ relative to part $j$. "Each
term in the sum can be interpreted as a spring that introduces spatial
constraints between a pair of parts, where the parameters $(a,b,c,d)$ specify
the rest location and rigidity of each spring." \citep{zhu2012face}
The $\alpha^m$ term refers to the bias of viewpoint mixture $m$. 

The score from equation \ref{eq:spec:fd:S} was used to determine which face 
detection results were the most reliable and suitable for this project.

The objective of the learning stage is to maximise $S(I,L,m)$ in Equation
\ref{eq:spec:fd:S} over $L$ and $m$, i.e:
\begin{equation}
\label{eq:spec:fd:maxS}
S^*(I) = \max_m[\max_L S(I,L,m)]
\end{equation}

Equation \ref{eq:spec:fd:maxS} says that for every mixture/viewpoint we find
the best set of part/facial landmark locations $L$ as scored by Equation
\ref{eq:spec:fd:S} and then choose the mixture with the highest score received.
Maximisation over $L$ is performed using dynamic programming.

\begin{itemize}
    \item Learning (?)
    \item Results comparison
\end{itemize}


\section{Preprocessing}
\label{spec:preproc}
\newcommand{\vleftright}{\vec{v}_i^{\text{left} \rightarrow \text{right}}}
\newcommand{\vi}[1]{\vec{v}_i^{\text{#1}}}
\newcommand{\vr}[1]{\vec{v}_{\text{ref}}^{\text{#1}}}
\newcommand{\vref}{\vec{v}_{\text{ref}}^{\text{left} \rightarrow \text{right}}}
The problem with using unprocessed Tinder profile pictures is that they are 
taken at a variety of different angles and scales. Before these pictures can 
be used to train a classifier they must first be transformed so that they all 
have the same angle and scale. This was done by using previously detected 
locations of eyes as reference points so that they always appear in the same 
location in the image. 

Throughout this section $\vi{left}$ and $\vi{right}$ will refer to the
coordinates of the left and right eyes, respectively, in the original image.
Vector $\vleftright$ passes through the locations of detected eyes in the image
and is defined as $\vleftright = \vi{right} - \vi{left}$. Vectors
$\vr{left}$ and $\vr{right}$ are our target eye locations to which all images
will be mapped.

Computing a rotation matrix is the first step in performing an affine transform
to ensure that all faces have the same rotation and scale. Before it can be 
computed we must first calculate the rotation angle $\theta$, scaling factor
$s$ and centre of the rotation $\vec{c}$ in the original image.

\begin{equation}
    \label{eq:spec:preproc:angle}
    \theta = -\arccos{\frac{\vleftright \cdot \vref}{|\vleftright| |\vref|}}
\end{equation}
Angle $\theta$ is rotation of the original face relative to the target face
location and can be calculated as the angle between vectors $\vref$ and
$\vleftright$ using the definiton of the dot product.

\begin{equation}
    \label{eq:spec:preproc:scales}
    s = \frac{|\vref|}{|\vleftright|}
\end{equation}

\begin{equation}
    \label{eq:spec:preproc:centre}
    \vec{c} = \frac{\vi{left} + \vi{right}}{2}
\end{equation}
The original image should be rotated around the midpoint of $\vi{left}$ and $\vi{right}$.

\begin{equation}
    \label{eq:spec:preproc:mat}
    \begin{bmatrix}
    \alpha & \beta  & (1-\alpha)\cdot \vec{c}_x - \beta \cdot \vec{c}_y + \vec{t}_x \\
    -\beta & \alpha & \beta \cdot \vec{c}_x + (1-\alpha)\cdot \vec{c}_y + \vec{t}_y
    \end{bmatrix}
\end{equation}
Using the computed rotation angle and point of rotation we obtain a 
transformation matrix (\ref{eq:spec:preproc:mat}) to perform this operation
using affine transform. In this equation $\alpha = s\cos\theta$ and $\beta =
s\sin\theta$. An additional translation vector ($\vec{t}$) was added to the third
column in this matrix to move the midpoint of the eyes in the image to the same
location as the midpoint of the reference eye locations (Equation
\ref{eq:spec:preproc:midtr}). 

\begin{equation}
    \label{eq:spec:preproc:midtr}
    \vec{t} = \frac{\vi{left} + \vi{right}}{2} - \frac{\vr{left} + \vr{right}}{2}
\end{equation}
After this transformation the distance between eyes and their locations
($\vi{left}$ and $\vi{right}$) should match the reference eye locations
($\vr{left}$ and $\vr{right}$).

\section{Feature extraction}


\section{Classification}

%%% Local Variables: 
%%% mode: latex
%%% TeX-master: "thesis"
%%% End: 
