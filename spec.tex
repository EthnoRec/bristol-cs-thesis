\chapter{Specification and system design}
\label{spec}
\section{Data collection}
The quantity and the quality of training data is crucial to all computer vision projects. 
However, for this task there was no luxury of using an existing dataset with 
the required level of detail in labels that would also be affordable.

A decision was made to collect own data for this project from a social network that 
makes it easy to associate user profiles with photos of themselves with locations accurate to city-level.
Popular social networks such as Facebook would be ideal for this, however it is not possible to obtain 
a list of profiles from a hand-picked location due to Facebook API constraints. Moreover, it is clearly 
stated in Facebook's Terms and Conditions that any way to circumvent these constraints would still count as a violation.


\subsection{Tinder}
Tinder is a mobile dating application that shows profiles closest to the user's GPS location. 
There have been numerous third-party Tinder utilities on the Web that reverse-engineered Tinder's simple 
and unobfuscated HTTP API that makes it possible to create a fully-featured Tinder client yourself.

The API has support for various actions such as liking and messaging users, but for the purposes of this project, 
the only relevant actions to us are authentication, setting own location using 
latitude and longitude coordinates and fetching nearby users.

\subsection{Authentication}
Tinder profiles are created using Facebook OAuth. When the application is opened for the first time by a non-registered user,
they will be prompted to login using Facebook OAuth which gives Tinder access to the user's full name, age, pictures and other details.

A user profile was created specifically for this project using the original Tinder mobile application on an Android phone. 

A Google Chrome extention was written in order to obtain an authentication token for the Tinder app on Facebook which must be
sent with every request to the Tinder API. The extension adds a button to the Google Chrome toolbar which opens a static webpage that
sends a request back to the extension to open the following URL:
\begin{logs}
https://www.facebook.com/v2.0/dialog/oauth
    ?response_type=token
    &display=popup
    &api_key=464891386855067
    &redirect_uri=fbconnect%3A%2F%2Fsuccess
    &scope=user_about_me%2Cuser_activities%2Cuser_education_history%2Cuser_location%2Cuser_photos%2Cuser_relationship_details%2Cuser_status'
\end{logs}
where \texttt{api\_key} specifies the Tinder application ID on Facebook and \texttt{scope} lists the type of information Tinder will gain access to.
This page asks the user to sign in with their Facebook credentials and, if successful, will save a cookie that will be sent along with the next request:

\begin{logs}
POST https://www.facebook.com/v2.0/dialog/oauth/confirm

{
  "app_id: "464891386855067",
  "ttstamp: "2658170904850115701205011500",
  "redirect_uri": "fbconnect://success",
  "return_format": "access_token",
  "from_post": 1,
  "display": "popup",
  "gdp_version": 4,
  "sheet_name": "initial",
  "__CONFIRM__": 1
}
\end{logs}
The response to this will contain an access token that will be used to interact with the Tinder API. It can be extracted 
using 
\begin{logs}
    access_token=([\w_]+)&
\end{logs} 
as a regular expression. It will be referred to as \texttt{<fb\_token>} from this point on.

Before we can use it to authenticate with Tinder the Facebook user ID must also be obtained by accessing \texttt{https://graph.facebook.com/me?access\_token=\textbf{<access\_token>}}.
The response to this request is a JSON object which contains the Facebook user ID under the property \texttt{id}, which will be known as \texttt{<fb\_user\_id>}.

%\begin{logs}
%POST https://api.gotinder.com/auth

%{
    %"facebook_token": fb_token, 
    %"facebook_id": fb_user_id
%}
%\end{logs}


%This made it possible to authenticate with Tinder using Facebook OAuth which takes Tinder's 
%application ID and upon successful Facebook authentication returns an authentication token that is sent with every Tinder API request.



\subsection{Updating location}
Updating the location of the user profile is simply a matter of sending the new latitude and longitude 
via a \texttt{POST} request to a profile change resource. The only limitation is that it is not possible to change the location too
frequently if the distance between the new and the old locations is too large to be travelled in the amount of time passed.

\subsection{Fetching profiles and images}

\subsection{Tinder job manager}
Tool to manage Tinder jobs.

\section{Face and landmark detection}

\section{Feature extraction}

\section{Classification}

%%% Local Variables: 
%%% mode: latex
%%% TeX-master: "thesis"
%%% End: 
