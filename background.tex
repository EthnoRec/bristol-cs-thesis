\chapter{Background and literature review}
\label{background}
There are many existing methods that have achieved extremely high accuracies
($95\%$ to $100\%$ \citep{muhammadg}) when classifying major races, such as
Caucasian and Asian, however the aim of this project is to go further and be
able to estimate potential country of birth. This is a far more challenging
task as different ethnic groups within the same race have significantly more
common facial features and visual similarities than ethnic groups from
different races.

\section{Literature review}
Most successful attempts at race qualification with 2 to 5 classes have relied
on local descriptor methods with the most common method being Local Binary
Patterns (LBP) (\citep{muhammadg}, \citep{yangLBP}, \citep{zhang2D3D}). Another
local descriptor method called Weber Local Descriptor (WLD) showed slightly
lower but comparable performance in a paper by \citep{muhammadg} and an
increase in accuracy through feature fusion by combining the histograms of LBP
and WLD.  

\subsection{Local Binary Patterns}
\label{bg:lbp}
The LBP operator works by selecting a pixel and using its value to threshold
neighbouring pixels that lie on a circle with that pixel as the centre. In most
implementations the neighbours can simply be 8 outside pixels of a $9 \times 9$ pixel
box with the target pixel in the centre.  The thresholded values are read
clockwise from 12 o’clock and form a binary number that is regarded as the LBP
value of that pixel. LBP values are calculated for all pixels in some image to
generate a histogram.  Usually, these histograms are generated in parallel by
performing on 16x16 pixel cells and then combining the histograms. These
histograms can then be used as features in classifiers.  


\subsection{Weber’s Law Descriptor}
This descriptor proposed by \citep{chenj} is based on a famous law in
psychology which states that the ratio of a change in stimulus to the original
stimulus is constant.  In WLD two metrics are computed for each pixel -
differential excitation and orientation.  To calculate the excitation the ratio
of the difference between a neighbouring pixel value and the target pixel value
to the target pixel value is found for every neighbour and then these ratios
are summed. Arctangent of the sum is then taken in order to avoid drastic
variation in the excitation values.  Orientation is simply the gradient angle
at the target pixel that is quantised to a dominant orientation. The number of
dominant orientations $T$ is selected manually. In the paper that proposed this
descriptor the value of 8 was used.  The excitation values are then grouped by
the dominant orientation and evenly split into $M$ bins, giving us a histogram
depicting the spread of excitation values for each orientation. Next, these $T$
histograms are further split into $M$ different histograms each, resulting in $T \times M$
histograms. These histograms are then grouped by the excitation bin and
sequentially concatenated to form a single histogram.  


\subsection{Holistic methods}
A Principal Component Analysis (PCA) approach to feature extraction was tested
on the same dataset as the fusion of previously mentioned LBP and WLD
histograms and resulted in average accuracy of $78.5\%$ from 5 classes with $96.83\%$
being the lowest class accuracy achieved. The LBP+WLD feature fusion led to
$99.06\%$ accuracy with $96.83\%$ lowest class accuracy.  

A more common holistic feature extraction method for this task is Linear
Discriminant Analysis (LDA) which is often used in combination with PCA to
reduce the dimension of the original pixel value vector for some image to $N-c$
where N is the number of training set instances and c is the number of classes.
This is required because LDA involves finding the inverse of within-class
scatter matrix which becomes singular (and thus not invertible) if the number
of training samples exceeds $N-c$.

%LDA calculations rely on finding the inverse of within-class scatter matrix
%which is a sum of matrices of rank 1 since every feature vector is centered
%around its own class mean and multiplied by the transpose of this centred
%feature vector. If the number of features exceeds the number of training
%instances then (?)

\improvements[inline,caption={}]{Review of \citep{chinesegroups} - recognition
of different Chinese ethnic groups (the most relevant paper to the aim of this
project).}


%%% Local Variables: 
%%% mode: latex
%%% TeX-master: "thesis"
%%% End: 
