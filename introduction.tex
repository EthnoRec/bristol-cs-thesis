\chapter{Introduction}
\label{introduction}

There have been many successful attempts at creating computer systems that can
perform the task of race recognition using images of human faces with striking
accuracy. One of the most useful papers for this project was \citep{muhammadg}
which reviews race classifiers using 2, 3 or 4 classes (major races) with
accuracies ranging from $0.94$ to $0.9906$.

The aim of this project was to go further and predict a much narrower ethnic
group of a person by analysing images of their face. This is a far more
challenging task as different ethnic groups within the same race have
significantly more common facial features and visual similarities than ethnic
groups from different races.

Surprisingly, there were not many studies done about this sub-class of the race
recognition problem.

\subsection{Biological aspects of race}
Race has been a significant issue throughout the history of humans 
and a source of many conflicts, however from a biological point of view
there are no distinct, pure races in humans, despite the vast diversity 
in visual traits of people from different parts of the world. \citep{onRace}

Nonetheless, we have always been keen to find patterns in nature and 
classify living beings. The number of major racial groups varies from 
one source to the other, but they are still recorded as a part of censuses 
in many countries to identify racial disparities in social and economic aspects 
\citep{censusRace}.

People living in multicultural countries such as the United States and the 
United Kingdom can often distinguish between people whose origins trace to other
parts of the world if their ancestors have not mixed with other races.

For example, it would be quite trivial for them to correctly identify if some
person originates from Africa, East Asia or Europe. This can be done just by
looking and analysing the person's face as it contains many important cues such
as skin colour, shape of the nose, distance between the eyes and many others.

%%% Local Variables: 
%%% mode: latex
%%% TeX-master: "thesis"
%%% End: 
