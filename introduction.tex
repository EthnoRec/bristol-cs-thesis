\chapter{Introduction}
\label{introduction}
Intro placeholder.

\section{Popular datasets for race recognition}
\label{introduction:datasets}
General discussion of popular race recognition datasets.
\subsection{Example}
\label{introduction:datasets:example}
Dataset called Example.

\section{Choosing a dataset for this experiment}
\label{introduction:ourdataset} 
Our dataset placeholder.

\section{Document Structure}
\label{introduction:structure} 
Structure placeholder.
%This document is structured as
%follows. Chapter~\ref{detection} introduces the \ac{ID}, that is the
%topic of our research. In particular, Chapter~\ref{detection}
%rigorously describes all the basic components that are necessary to
%define the \ac{ID} task and an \acp{IDS}. The reader with knowledge on
%this subject may skip the first part of the chapter and focus on
%Section~\ref{detection:ad} and \ref{detection:correlation} that
%include a comprehensive review of the most relevant and influential
%modern approaches on network-, host-, web-based \ac{ID} techniques,
%along with a separate overview of the alert correlation approaches.

%As described in Section~\ref{introduction:contributions}, the
%description of our contributions is structured into three
%chapters. Chapter~\ref{host} focuses on host-based techniques,
%Chapter~\ref{web} regards web-based anomaly detection, while
%Chapter~\ref{correlation} described two techniques that allow to
%recognize relations between alerts reported by network- and host-based
%systems. Reading Section~\ref{detection:ad:network} is recommended
%before reading Chapter~\ref{correlation}.

%The reader interested in protection techniques for the operating
%system can skim through Section~\ref{detection:ad:host} and then read
%Chapter~\ref{host}. The reader with interests on web-based protection
%techniques can read Section~\ref{detection:ad:web} and then
%Chapter~\ref{web}. Similarly, the reader interested in alert
%correlation systems can skim through
%Section~\ref{detection:ad:network} and \ref{detection:ad:host} and
%then read Chapter~\ref{correlation}.

%%% Local Variables: 
%%% mode: latex
%%% TeX-master: "thesis"
%%% End: 
