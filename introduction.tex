\chapter{Introduction}
\label{introduction}

% Biological aspects
Race has been a significant issue throughout the history of humans 
and a source of many conflicts, however from a biological point of view
there are no distinct, pure races in humans, despite the vast diversity 
in visual traits of people from different parts of the world.
% cite: http://www.physanth.org/about/position-statements/biological-aspects-race/

Nonetheless, we have always been keen to find patterns in nature and 
classify living beings. The number of major racial groups varies from 
one source to the other, but they are still recorded as a part of censuses 
in many countries to identify racial disparities in social and economic aspects.
% cite: http://www.census.gov/topics/population/race/about.html

People living in multicultural countries such as the United States and the 
United Kingdom can often distinguish between people whose origins trace to other
parts of the world if their ancestors have not mixed with other races.

For example, it would be quite trivial for them to correctly identify if some person
originates from Africa, East Asia or Europe. This can be done just by looking and
analysing the person's face as it contains many important cues such as skin colour, 
shape of the nose, distance between the eyes and many others.

There have been many successful attempts at creating computer systems that can perform
this task using images of human faces with striking accuracy. One of the 
most useful papers for this project was \citep{muhammadg} which reviews 
race classifiers using 2, 3 or 4 classes (major races) with accuracies 
ranging from \textbf{X} to \textbf{Y}.

The aim of this project was to go further and predict a much narrower ethnic group 
of a person by analysing images of their face. This is a far more challenging task 
as different ethnic groups within the same race have significantly more common 
facial features and visual similarities than ethnic groups from different races.

Surprisingly, there were not many studies done about this sub-class of the race 
recognition problem.

\improvements[inline,caption={}]{I can briefly introduce a paper with a similar
goal \citep{chinesegroups}. Should I avoid technical details in this section?}

\section{Document Structure}
\label{introduction:structure} 
Structure placeholder.
%This document is structured as
%follows. Chapter~\ref{detection} introduces the \ac{ID}, that is the
%topic of our research. In particular, Chapter~\ref{detection}
%rigorously describes all the basic components that are necessary to
%define the \ac{ID} task and an \acp{IDS}. The reader with knowledge on
%this subject may skip the first part of the chapter and focus on
%Section~\ref{detection:ad} and \ref{detection:correlation} that
%include a comprehensive review of the most relevant and influential
%modern approaches on network-, host-, web-based \ac{ID} techniques,
%along with a separate overview of the alert correlation approaches.

%As described in Section~\ref{introduction:contributions}, the
%description of our contributions is structured into three
%chapters. Chapter~\ref{host} focuses on host-based techniques,
%Chapter~\ref{web} regards web-based anomaly detection, while
%Chapter~\ref{correlation} described two techniques that allow to
%recognize relations between alerts reported by network- and host-based
%systems. Reading Section~\ref{detection:ad:network} is recommended
%before reading Chapter~\ref{correlation}.

%The reader interested in protection techniques for the operating
%system can skim through Section~\ref{detection:ad:host} and then read
%Chapter~\ref{host}. The reader with interests on web-based protection
%techniques can read Section~\ref{detection:ad:web} and then
%Chapter~\ref{web}. Similarly, the reader interested in alert
%correlation systems can skim through
%Section~\ref{detection:ad:network} and \ref{detection:ad:host} and
%then read Chapter~\ref{correlation}.

%%% Local Variables: 
%%% mode: latex
%%% TeX-master: "thesis"
%%% End: 
